\documentclass[UTF8]{ctexart}
\usepackage{geometry}
\usepackage{indentfirst}
\usepackage{hyperref}
\usepackage{harpoon}
\usepackage{amsmath}
\usepackage{graphicx}
\usepackage{float}
\usepackage{subfigure}
\usepackage{multirow}
\usepackage{array}
\usepackage{tikz}
\usetikzlibrary{arrows, shapes, positioning, calc}
\geometry{a4paper, left=1cm, right=1cm, top=2cm, bottom=2cm}
\setlength{\parindent}{1cm}
\renewcommand\contentsname{Content}
\title{Report 1: Methods to Compute Pseudopotential}
\author{Yuanxing Duan 段元兴}
\date{\today}
\begin{document}
\maketitle
\thispagestyle{empty}
\setcounter{page}{1}
\newpage
\tableofcontents
\newpage
    \section{Introduction to pseudopotential}
        \indent A pseudopotential (effective potential) is used as an approximation for the
        simplified description of complex systems. For example, replacing the complex core elections
        and nucleus inside an atomic with a simplified potential $V_{pseudo}$ so that Schrödinger
        equation contains a modified effective potential term instead of the Coulombic potential
        term for core electrons normally found in the Schrödinger equation. In this way the core
        states are eliminated and the valence elections are described by pseudo-wavefunctions $\psi_{pseudo}$
        with significantly fewer nodes. And because the core electrons are usually more local, so require more
        higher energy plane waves in the basis set, which means higher cutoff energy $E_{cut}=\dfrac{\hbar^2G_{cut}^2}{2m}$.
        With pseudopotential we can ignore these so valence elections + pseudopotential are much more efficient than
        considering all electrions.\\
        First-principles pseudopotentials are derived from an atomic reference state, requiring that the pseudo and all
        electron valence eigenstates have the same energies and amplitude (and thus density) outside a chosen core cut-off
        radius $r_c$. And thus there are two approximation:\\
        \indent 1. It's a picture of one election,\\
        \indent 2. The small-core approximation assumes that there is no significant overlap between core and valence wave-function.\\
    \section{Pseudopotentials}
        \indent Norm-conserving and ultrasoft are the two most common forms of pseudopotential used in modern plane-wave electronic
        structure codes. They allow a basis-set with a significantly lower cut-off (the frequency of the highest Fourier mode) to be
        used to describe the electron wavefunctions and so allow proper numerical convergence with reasonable computing resources.
        \subsection{Norm-conserving pseudopotential}
        \indent Norm-conserving pseudopotential was first proposed by Hamann, Schlüter, and Chiang (HSC) in 1979. The original HSC
        norm-conserving pseudopotential takes the following form:
        \begin{equation}
            \hat{V}_{ps}(r)=\sum _{l}\sum _{m}|Y_{lm}\rangle V_{lm}(r)\langle Y_{lm}|
        \end{equation}
        where $|Y_{lm}\rangle$ projects a one-particle wavefunction, such as one Kohn-Sham orbital, to the angular momentum labeled by
        $\{l,m\}$. $V_{lm}(r)$ is the pseudopotential that acts on the projected component. Different angular momentum states then feel
        different potentials, thus the HSC norm-conserving pseudopotential is non-local, in contrast to local pseudopotential which acts
        on all one-particle wave-functions in the same way. Norm-conserving pseudopotentials are constructed to enforce two conditions:\\
        \indent 1. Inside the cut-off radius $r_c$, the norm of each pseudo-wavefunction be identical to its corresponding all-electron wavefunction:
        \begin{equation}
            \int_0^{r_c}dr^3\phi_{\mathbf{R},i}({\vec{r}})\phi_{\mathbf{R},j}({\vec{r}})=
            \int_0^{r_c}dr^{3}{\tilde{\phi}}_{\mathbf{R},i}({\vec{r}}){\tilde{\phi}}_{\mathbf{R},j}({\vec{r}})
        \end{equation}
        where $\phi_{\mathbf{R},i}$ and $\tilde{\phi}_{\mathbf{R},i}$ are the all-electron and pseudo reference states for the pseudopotential
        on atom $\mathbf{R}$.\\
        \indent 2. All-election and pseudo wavefunctions are identical outside cutoff radius $r_c$.
\end{document}